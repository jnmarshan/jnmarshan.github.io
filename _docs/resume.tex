% Place name at left
{\huge \name}

% Alternatively, print name centered and bold:
%\centerline{\huge \bf \name}

\vspace{0.25in}

\begin{minipage}{0.45\linewidth}
  Unit 5381 Warrumbul Lodge\\
  4 Hutton Street\\
  Acton ACT 2601, Australia
\end{minipage}
\begin{minipage}{0.45\linewidth}
  \begin{tabular}{ll}
    Phone: +61 452 765 881 (mobile)  \\
    Email: \href{mailto:joseph.marshan@anu.edu.au}{\tt joseph.marshan@anu.edu.au} \\
    Homepage:  \href{https://www.rse.anu.edu.au/about-us/our-people/people/?profile=Joseph-Marshan}{\tt RSE staff website} \\
  \end{tabular}
\end{minipage}


\section*{Personal}

\begin{itemize}
\item Born on September 8, 1989
\item Indonesian citizen. Australian Visa Subclass 573
\end{itemize}

\section*{Research Interests}
\begin{itemize}
	\item Primary: Labor economics, female labor force participation
	\item Secondary: Poverty, family economics, education
\end{itemize}


\section*{Education}

\begin{itemize}
  \item Ph.D. candidate in Economics, Research School of Economics, The Australian National University, 2017 - 2021 (expected).

  \item M.Ec. (Economics), Research School of Economics, The Australian National University, 2015.

  \item B.A. in Economics, Department of Economics, Universitas Indonesia, 2011.

\end{itemize}


\section*{Employments}

\begin{itemize}
\item Research Assistant John Mitchell Economics of Poverty Lab, November 2019 - present
\item Teaching Assistant, The Australian National University, July 2017 - present
\item Researcher, SMERU Research Institute, February 2016 - February 2017
\item Junior Researcher, SMERU Research Institute, December 2012 - January 2014
\item Research Assistant, Harvard Kennedy School Indonesia Program, September 2011 - December 2012
\item Teaching Assistant, Universitas Indonesia, August 2008 - June 2011
\end{itemize}

\section*{Teaching Assistants}

\begin{itemize}
\item Econometric Methods and Modelling, Postgraduate level, Semester 1 - 2020
\item Case Studies in Applied Economic Analysis and Econometrics, Postgraduate level, Semester 2 - 2019
\item Econometric Methods and Modelling, Postgraduate level, Semester 1 - 2019
\item Applied Micro-econometrics, Postgraduate level, Semester 2 - 2018
\item Microeconomics 2, Undergraduate level, Semester 2 - 2017

\end{itemize}

\section*{Working Papers}

\begin{itemize}
\item 
Intergenerational link of female labor force participation: Evidence from Indonesia \\

Abstract: \\
I investigate the existence of intergenerational link of female labor force participation in Indonesia using a rich large-scale longitudinal data known as Indonesia Family Life Survey (IFLS). The IFLS gives opportunity to draw intergenerational link between mother's and daughter's labor force participation. This study contributes to limited empirical evidence on intergenerational link in female labor market in a developing country setup. I find that such intergenerational link exist, in particular, for those who live in urban area. Mother's participation effect, albeit small, is four times larger in size than the effect of additional one year of education to daughter's labor force participation. This suggests meaningful and the importance of intergenerational belief transfer in shaping the next generation of female labor force. I also find that community that adopts pro-gender norms preserve and strengthen the intergenerational link. Finally, I provide evidence that transfer of gender role belief from mother to daughter as the plausible mechanism.

 

\item 
Quality-quantity trade-off: an evidence from Indonesia. \\

Abstract: \\
This paper provides evidence on quantity and quality trade-off (Becker and Lewis, 1973) existence in Indonesia using variation of province-level total fertility rate by cohorts as an instrument variable. I use the fact that Indonesia's nationwide family planning policy, known as Keluarga Berencana, became ineffective
due to weaker implementation, after an exogenous political shock in 1998 that dramatically turn Indonesia into more decentralized regime. This event unexpectedly have plateaued total fertility rate in Indonesia post-1998. Using the fourth wave Indonesia Family Life Survey data, I investigate the trade-off on
children aged less than 15 years old who are born between 1992 and 2006. As contribution to previous efforts (Millimet and Wang, 2011), this paper also investigate such trade-off separately for urban and rural settings. I find trade-off exists only for height-for-age in rural areas. It does exists for year of schooling and has greater impact for urban samples. These results support the argument of potential role of public goods provision in reducing the cost of trade-off (Angrist et al., 2010). My findings robust over alternative instrument variable constructions, birth order effect, and falsification test. The results support rising concern on the need of revitalization of family planning policy in Indonesia.
\end{itemize}


\section*{Chapters}

\begin{itemize}
\item 
Structural Transformation and the Release of Labor from Agriculture. In {\it Indonesia: Enhancing Productivity Through Quality Jobs} 100--129. Asian Development Bank. February 20018
\end{itemize}


\subsection*{Research Reports}
\begin{itemize}
\item 
\item The Well-Being of Poor Children Left by Their Mothers who Become Migrant Workers: Case Study in Two Kabupaten in Indonesia {\it SMERU Research Report}. May 2017
\item The Dynamics of Poor Women Livelihood: A Case Study amidst a Fuel Price Change {\it SMERU Research Report}. December 2016.
\item Prevalence of Child Marriage and Its
Determinant among Young Woman in Indonesia. Co-author with Fajar Rakhmadi, \& Mayang Rizky {\it SMERU Research Report}. November 2015.
\end{itemize}

\section*{Awards}

\begin{itemize}
\item Chris Higgins Prize 2016. Best research paper in Case Studies in Applied Economic Analysis and Econometrics.
\item LPDP Scholarship 2014-2016, 2017-2021. Master and Doctoral program.
\end{itemize}

\section*{Skills}
\begin{itemize}
\item Proficient in STATA
\item Working knowledge in \LaTeX, QGIS, Python, and MatLab.
\end{itemize}

\section*{Languages}
\begin{itemize}
\item Indonesia (Native)
\item English (Professional)
\item Spanish (Basic)
\end{itemize}

\section*{References}
\begin{itemize}
	\item Xin Meng  (Thesis' supervisor). \\
	The Australian National University. \\
	Email:  \href{mailto:xin.meng@anu.edu.au}{\tt xin.meng@anu.edu.au}

	\item Asep Suryahadi (Former employer). \\
	SMERU Research Institute.  \\
	Email: \href{mailto:suryahadi@smeru.or.id}{\tt suryahadi@smeru.or.id}

	\item Dana Hana (RSE TA Coordinator). \\
	The Australian National University. \\
	Email: \href{mailto:dana.hanna@anu.edu.au}{\tt dana.hanna@anu.edu.au}
\end{itemize}

\bigskip

% Footer
\begin{center}
  \begin{footnotesize}
    Last updated: January 2020 \\
    %\href{\footerlink}{\texttt{\footerlink}}
  \end{footnotesize}
\end{center}

